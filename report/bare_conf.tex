\documentclass[conference]{IEEEtran}
\usepackage{fullpage} % Package to use full page
\usepackage{parskip} % Package to tweak paragraph skipping
\usepackage{tikz} % Package for drawing
\usepackage{amsmath}
\usepackage{hyperref}
\usepackage{listings}
\usepackage{graphicx}
\usepackage{caption}
\usepackage{subcaption}
\usepackage{romannum}
\usepackage{amsmath}
\DeclareMathOperator*{\argmax}{arg\,max}
% Add the compsoc option for Computer Society conferences.
%
% If IEEEtran.cls has not been installed into the LaTeX system files,
% manually specify the path to it like:
% \documentclass[conference]{../sty/IEEEtran}





% Some very useful LaTeX packages include:
% (uncomment the ones you want to load)


% *** MISC UTILITY PACKAGES ***
%
%\usepackage{ifpdf}
% Heiko Oberdiek's ifpdf.sty is very useful if you need conditional
% compilation based on whether the output is pdf or dvi.
% usage:
% \ifpdf
%   % pdf code
% \else
%   % dvi code
% \fi
% The latest version of ifpdf.sty can be obtained from:
% http://www.ctan.org/tex-archive/macros/latex/contrib/oberdiek/
% Also, note that IEEEtran.cls V1.7 and later provides a builtin
% \ifCLASSINFOpdf conditional that works the same way.
% When switching from latex to pdflatex and vice-versa, the compiler may
% have to be run twice to clear warning/error messages.






% *** CITATION PACKAGES ***
%
%\usepackage{cite}
% cite.sty was written by Donald Arseneau
% V1.6 and later of IEEEtran pre-defines the format of the cite.sty package
% \cite{} output to follow that of IEEE. Loading the cite package will
% result in citation numbers being automatically sorted and properly
% "compressed/ranged". e.g., [1], [9], [2], [7], [5], [6] without using
% cite.sty will become [1], [2], [5]--[7], [9] using cite.sty. cite.sty's
% \cite will automatically add leading space, if needed. Use cite.sty's
% noadjust option (cite.sty V3.8 and later) if you want to turn this off.
% cite.sty is already installed on most LaTeX systems. Be sure and use
% version 4.0 (2003-05-27) and later if using hyperref.sty. cite.sty does
% not currently provide for hyperlinked citations.
% The latest version can be obtained at:
% http://www.ctan.org/tex-archive/macros/latex/contrib/cite/
% The documentation is contained in the cite.sty file itself.






% *** GRAPHICS RELATED PACKAGES ***
%
\ifCLASSINFOpdf
  % \usepackage[pdftex]{graphicx}
  % declare the path(s) where your graphic files are
  % \graphicspath{{../pdf/}{../jpeg/}}
  % and their extensions so you won't have to specify these with
  % every instance of \includegraphics
  % \DeclareGraphicsExtensions{.pdf,.jpeg,.png}
\else
  % or other class option (dvipsone, dvipdf, if not using dvips). graphicx
  % will default to the driver specified in the system graphics.cfg if no
  % driver is specified.
  % \usepackage[dvips]{graphicx}
  % declare the path(s) where your graphic files are
  % \graphicspath{{../eps/}}
  % and their extensions so you won't have to specify these with
  % every instance of \includegraphics
  % \DeclareGraphicsExtensions{.eps}
\fi
% graphicx was written by David Carlisle and Sebastian Rahtz. It is
% required if you want graphics, photos, etc. graphicx.sty is already
% installed on most LaTeX systems. The latest version and documentation can
% be obtained at: 
% http://www.ctan.org/tex-archive/macros/latex/required/graphics/
% Another good source of documentation is "Using Imported Graphics in
% LaTeX2e" by Keith Reckdahl which can be found as epslatex.ps or
% epslatex.pdf at: http://www.ctan.org/tex-archive/info/
%
% latex, and pdflatex in dvi mode, support graphics in encapsulated
% postscript (.eps) format. pdflatex in pdf mode supports graphics
% in .pdf, .jpeg, .png and .mps (metapost) formats. Users should ensure
% that all non-photo figures use a vector format (.eps, .pdf, .mps) and
% not a bitmapped formats (.jpeg, .png). IEEE frowns on bitmapped formats
% which can result in "jaggedy"/blurry rendering of lines and letters as
% well as large increases in file sizes.
%
% You can find documentation about the pdfTeX application at:
% http://www.tug.org/applications/pdftex





% *** MATH PACKAGES ***
%
%\usepackage[cmex10]{amsmath}
% A popular package from the American Mathematical Society that provides
% many useful and powerful commands for dealing with mathematics. If using
% it, be sure to load this package with the cmex10 option to ensure that
% only type 1 fonts will utilized at all point sizes. Without this option,
% it is possible that some math symbols, particularly those within
% footnotes, will be rendered in bitmap form which will result in a
% document that can not be IEEE Xplore compliant!
%
% Also, note that the amsmath package sets \interdisplaylinepenalty to 10000
% thus preventing page breaks from occurring within multiline equations. Use:
%\interdisplaylinepenalty=2500
% after loading amsmath to restore such page breaks as IEEEtran.cls normally
% does. amsmath.sty is already installed on most LaTeX systems. The latest
% version and documentation can be obtained at:
% http://www.ctan.org/tex-archive/macros/latex/required/amslatex/math/





% *** SPECIALIZED LIST PACKAGES ***
%
%\usepackage{algorithmic}
% algorithmic.sty was written by Peter Williams and Rogerio Brito.
% This package provides an algorithmic environment fo describing algorithms.
% You can use the algorithmic environment in-text or within a figure
% environment to provide for a floating algorithm. Do NOT use the algorithm
% floating environment provided by algorithm.sty (by the same authors) or
% algorithm2e.sty (by Christophe Fiorio) as IEEE does not use dedicated
% algorithm float types and packages that provide these will not provide
% correct IEEE style captions. The latest version and documentation of
% algorithmic.sty can be obtained at:
% http://www.ctan.org/tex-archive/macros/latex/contrib/algorithms/
% There is also a support site at:
% http://algorithms.berlios.de/index.html
% Also of interest may be the (relatively newer and more customizable)
% algorithmicx.sty package by Szasz Janos:
% http://www.ctan.org/tex-archive/macros/latex/contrib/algorithmicx/




% *** ALIGNMENT PACKAGES ***
%
%\usepackage{array}
% Frank Mittelbach's and David Carlisle's array.sty patches and improves
% the standard LaTeX2e array and tabular environments to provide better
% appearance and additional user controls. As the default LaTeX2e table
% generation code is lacking to the point of almost being broken with
% respect to the quality of the end results, all users are strongly
% advised to use an enhanced (at the very least that provided by array.sty)
% set of table tools. array.sty is already installed on most systems. The
% latest version and documentation can be obtained at:
% http://www.ctan.org/tex-archive/macros/latex/required/tools/


%\usepackage{mdwmath}
%\usepackage{mdwtab}
% Also highly recommended is Mark Wooding's extremely powerful MDW tools,
% especially mdwmath.sty and mdwtab.sty which are used to format equations
% and tables, respectively. The MDWtools set is already installed on most
% LaTeX systems. The lastest version and documentation is available at:
% http://www.ctan.org/tex-archive/macros/latex/contrib/mdwtools/


% IEEEtran contains the IEEEeqnarray family of commands that can be used to
% generate multiline equations as well as matrices, tables, etc., of high
% quality.


%\usepackage{eqparbox}
% Also of notable interest is Scott Pakin's eqparbox package for creating
% (automatically sized) equal width boxes - aka "natural width parboxes".
% Available at:
% http://www.ctan.org/tex-archive/macros/latex/contrib/eqparbox/





% *** SUBFIGURE PACKAGES ***
%\usepackage[tight,footnotesize]{subfigure}
% subfigure.sty was written by Steven Douglas Cochran. This package makes it
% easy to put subfigures in your figures. e.g., "Figure 1a and 1b". For IEEE
% work, it is a good idea to load it with the tight package option to reduce
% the amount of white space around the subfigures. subfigure.sty is already
% installed on most LaTeX systems. The latest version and documentation can
% be obtained at:
% http://www.ctan.org/tex-archive/obsolete/macros/latex/contrib/subfigure/
% subfigure.sty has been superceeded by subfig.sty.



%\usepackage[caption=false]{caption}
%\usepackage[font=footnotesize]{subfig}
% subfig.sty, also written by Steven Douglas Cochran, is the modern
% replacement for subfigure.sty. However, subfig.sty requires and
% automatically loads Axel Sommerfeldt's caption.sty which will override
% IEEEtran.cls handling of captions and this will result in nonIEEE style
% figure/table captions. To prevent this problem, be sure and preload
% caption.sty with its "caption=false" package option. This is will preserve
% IEEEtran.cls handing of captions. Version 1.3 (2005/06/28) and later 
% (recommended due to many improvements over 1.2) of subfig.sty supports
% the caption=false option directly:
%\usepackage[caption=false,font=footnotesize]{subfig}
%
% The latest version and documentation can be obtained at:
% http://www.ctan.org/tex-archive/macros/latex/contrib/subfig/
% The latest version and documentation of caption.sty can be obtained at:
% http://www.ctan.org/tex-archive/macros/latex/contrib/caption/




% *** FLOAT PACKAGES ***
%
%\usepackage{fixltx2e}
% fixltx2e, the successor to the earlier fix2col.sty, was written by
% Frank Mittelbach and David Carlisle. This package corrects a few problems
% in the LaTeX2e kernel, the most notable of which is that in current
% LaTeX2e releases, the ordering of single and double column floats is not
% guaranteed to be preserved. Thus, an unpatched LaTeX2e can allow a
% single column figure to be placed prior to an earlier double column
% figure. The latest version and documentation can be found at:
% http://www.ctan.org/tex-archive/macros/latex/base/



%\usepackage{stfloats}
% stfloats.sty was written by Sigitas Tolusis. This package gives LaTeX2e
% the ability to do double column floats at the bottom of the page as well
% as the top. (e.g., "\begin{figure*}[!b]" is not normally possible in
% LaTeX2e). It also provides a command:
%\fnbelowfloat
% to enable the placement of footnotes below bottom floats (the standard
% LaTeX2e kernel puts them above bottom floats). This is an invasive package
% which rewrites many portions of the LaTeX2e float routines. It may not work
% with other packages that modify the LaTeX2e float routines. The latest
% version and documentation can be obtained at:
% http://www.ctan.org/tex-archive/macros/latex/contrib/sttools/
% Documentation is contained in the stfloats.sty comments as well as in the
% presfull.pdf file. Do not use the stfloats baselinefloat ability as IEEE
% does not allow \baselineskip to stretch. Authors submitting work to the
% IEEE should note that IEEE rarely uses double column equations and
% that authors should try to avoid such use. Do not be tempted to use the
% cuted.sty or midfloat.sty packages (also by Sigitas Tolusis) as IEEE does
% not format its papers in such ways.





% *** PDF, URL AND HYPERLINK PACKAGES ***
%
%\usepackage{url}
% url.sty was written by Donald Arseneau. It provides better support for
% handling and breaking URLs. url.sty is already installed on most LaTeX
% systems. The latest version can be obtained at:
% http://www.ctan.org/tex-archive/macros/latex/contrib/misc/
% Read the url.sty source comments for usage information. Basically,
% \url{my_url_here}.





% *** Do not adjust lengths that control margins, column widths, etc. ***
% *** Do not use packages that alter fonts (such as pslatex).         ***
% There should be no need to do such things with IEEEtran.cls V1.6 and later.
% (Unless specifically asked to do so by the journal or conference you plan
% to submit to, of course. )


% correct bad hyphenation here
\hyphenation{op-tical net-works semi-conduc-tor}


\begin{document}

\title{DL}


\author{\IEEEauthorblockN{ZHIHAN WANG}
\IEEEauthorblockA{University of Southampton\\
School of Electronic and\\Computer Science\\
Email: zw3u18@soton.ac.uk}
\and
\IEEEauthorblockN{ALEX}
\IEEEauthorblockA{University of Southampton\\
School of Electronic and\\Computer Science\\
Email: zw3u18@soton.ac.uk}
\and
\IEEEauthorblockN{SHAUNAK}
\IEEEauthorblockA{University of Southampton\\
School of Electronic and\\Computer Science\\
Email: zw3u18@soton.ac.uk}
}


% make the title area
\maketitle


\begin{abstract}
%\boldmath
Abstract goes here.The paper we are going to reproduce is~\cite{street} 

\Romannum{2}.  part \Romannum{3} 






\end{abstract}
% IEEEtran.cls defaults to using nonbold math in the Abstract.
% This preserves the distinction between vectors and scalars. However,
% if the conference you are submitting to favors bold math in the abstract,
% then you can use LaTeX's standard command \boldmath at the very start
% of the abstract to achieve this. Many IEEE journals/conferences frown on
% math in the abstract anyway.

% no keywords




% For peer review papers, you can put extra information on the cover
% page as needed:
% \ifCLASSOPTIONpeerreview
% \begin{center} \bfseries EDICS Category: 3-BBND \end{center}
% \fi
%
% For peerreview papers, this IEEEtran command inserts a page break and
% creates the second title. It will be ignored for other modes.
\IEEEpeerreviewmaketitle



\section{Introduction}
% no \IEEEPARstart
Optical character recognition has been well studied on constrained domains, such as document processing, but is still challenging in unconstrained domains, such as natural photographs. In the paper~\cite{street}, an equally hard sub-problem, arbitrary multi-character text recognition in photos captured at street level, has been discussed.

The paper employed DistBelief, a software framework that can utilize computing clusters with thousands of machines to train large models\cite{disblief} to implement large-scale deep neural networks on publicly available Street View House Numbers(SVHN) dataset and finally achieved over 96\% accuracy in recognizing street numbers, and 97.84\% accuracy on per-digit recognition tasks. After then, this trained model was implemented to solve CAPTCHA puzzles where text is deliberatly distorted and used to distinguish humans and robots, and achieved a 99.8\% accuracy.

\cite{street} contributes a lot and the results even reached human level performance at specific thresholds. The detailed information goes below.

% Problems descriptions (backgroud, 2 problem targets, methods outline, results)
% er be an issue)

\subsection{Architecture}
There are three steps in
traditional approaches to recognize multi-digit numbers from photos, localization, segmentation, and recognition. 
The paper~\cite{street} proposed a unified model to integrate these three steps via the use of a deep convolutional neural network that operates directly on the image pixels and achieved an end-to-end prediction.
\subsubsection{Basic methods}
The task of street number recognition is that given an image, the numbers in the image should be identified. The basic method used here is to train a probabilistic model of a predicted sequence output given an image. Let \textbf{S} represent the output sequence and $X$ represent the input image. The goal is to learn a model of $P(\textbf{S}|X)$ by maximizing $\log P(\textbf{S}|X)$ on the training dataset. The probability of a specific sequences \textbf{s}=$s_1,s_2,...,s_n$ is given as below, in which $n$ is the number of digits in the image.
\begin{equation*}
    P(\textbf{S}=\textbf{s}|X)=P(L=n|X)\prod_{i=1}^nP(S_i=s_i|X)
\end{equation*}
At prediction time,
\begin{equation*}
    \textbf{s}=(l,s_1,s_2,...,s_l)=\argmax_{L,S_1,S_2,...,S_L}\log P(S|X)
\end{equation*}
\subsubsection{CNN structure}
The best model trained on the SVHM dataset in \cite{street} is with 11 hidden layers, consisting of 8 convolutional layers, 1 locally connected hidden layers and 2 densly connected hidden layers. All connections are feedforward and there are not skipped layers. The first hidden layer contains maxout units\cite{maxout} and each unit is with three filters while other layers contain ReLU. Each convolutional layer includes max pooling with window size $2\times2$ and subtractive normalizaiton with window size $3\times3$.
The stride at each layer alternates between 1 and 2; therefore, zero padding is used to preserve representation size. The size of all the kernels is in $5\times5$. As significant overfitting can be seen, dropout applied to all hidden layers.
\subsection{Performance}
\cite{street} shows that the performance of model increases with the depth of the convolutional network. Two experiments were used to prove this conclusion. The first one confirmed that depth is necessary for good performance and the second one with a accuracy graph demonstrated that smaller models even with more parameters cannot reach the same level of the performance as deeper models.

\subsection{Comparison with previous work}
Images recognition networks trained in the previously published papers generally have 2 to 4 convolutional layers followed by 1 or 2 densely connected layers and classification layers. But the model in~\cite{street} used more convolutional layers as referred above. This is because earlier layers are used to solve localizaiton and segmentation firstly, and then the results are processed to later layers to recognize. This model achieves an end-to-end prediction.
% \begin{figure}[h]
% \begin{center}
% \includegraphics[scale=0.3]{"latent".jpg}
% \end{center}
% % \caption{Comparison among 3 models. }
% \label{fig1}
% \end{figure}
%  ($L\rightarrow Y$) from the relationship ($X\rightarrow Y$) 


% \subsection{~~~~}







%\begin{table}[!t]
%% increase table row spacing, adjust to taste
%\renewcommand{\arraystretch}{1.3}
% if using array.sty, it might be a good idea to tweak the value of
% \extrarowheight as needed to properly center the text within the cells
%\caption{An Example of a Table}
%\label{table_example}
%\centering
%% Some packages, such as MDW tools, offer better commands for making tables
%% than the plain LaTeX2e tabular which is used here.
%\begin{tabular}{|c||c|}
%\hline
%One & Two\\
%\hline
%Three & Four\\
%\hline
%\end{tabular}
%\end{table}

% command of the stfloats package.
\section{Reproduction}
In this part, we are going to reproduce section 5.1\cite{street} and show the results on the public Street View House Numbers dataset. The detailed information is described as below.
\subsection{SVHN dataset}
The SVHN dataset is obtained from \cite{image}.It contains 200k street numbers with the location information of individual digit in each image. 

In the data preprocessing, we find the rectangular bounding box that only contains an individual character firstly. Then we expand the bounding box in both $x$ and $y$ direction by 30\% to crop the image. Next, resize the output to $64\times64$ pixels. To increase the size of the dataset, we will randomly choose a location with a $54\times54$ window to crop this $64\times64$ image many times as this can give us multiple versions from one training image. 
\subsection{Architecture}
The model used to train is a Convolutional Neural Netwrok with 11 hidden layers. The first 8 hidden layers are convolutional ones followed by 2 densely connected hidden layers and 1 locally connected hidden one. Each convolutional layer is with a kernel filter in $5\times5$, a $2\times2$ window to achieve max pooling, a batch normalization unit and a generalization part which is a maxout unit in the first convolutional layer and ReLU in the left 7 layers. The architecture is shown in Fig\ref{fig1}.

\begin{figure}[h]
\centering
% \begin{subfigure}[b]{0.5\textwidth}%
% \includegraphics[width=0.75\linewidth]{"A48".jpg}
% \caption{Stock A48TKB; Strike price: 5600}
% \label{fig1:sub1}
% \end{subfigure}%
% \begin{subfigure}[b]{0.5\textwidth}%
% \includegraphics[width=0.75\linewidth]{"C22".jpg}
% \caption{Stock C22BMY; Strike price: 4800}
% \label{fig1:sub2}
% \end{subfigure}
% \begin{subfigure}[b]{0.5\textwidth}%
\includegraphics[width=0.75\linewidth]{"arch".jpeg}
\caption{Model architecture. The output of the first convolutional layer is $32\times27\times27\times48$ where 32 is batch size, 48 is the number of units at each spatial location, and 27 is the output dimension. The output dimension of convolutional layer is calculated by $O=\frac{I-K+2P}{S}+1$ where $I$ is the input dimension, $K$ is kernel size, $P$ is padding and $S$ is for strike. In the computation of the output of the first layer, $I=54, K=5, P=2, S=2$. And after max pooling, the dimension is calculated by $O=\frac{I-P}{S}+1$ using the parameter value from last step. They are $I=27, P=1, S=1$ respectively, so the final output dimension is 27. The red number part shows after the convolution and max pooling, the output matrix is required to be flattened to be proceeded into the densely connected layers.   }
% \label{fig1:sub3}
% \end{subfigure}

\label{fig1}
\end{figure}
To achieve the maxout unit, we used 3 filters to project the original matrix seperately and only keep the maximum as the output. In addition, the stride used at each convolutional layer alternates between 2 and 1. Therefore, half the convolutional layers do not reduce the representation size. We also use zero padding in all the convolutional layers to keep spatial size of representation. To avoid the overfitting case, dropout units are used in the training part.

Another interesting point is that we use a parameter named $coverage$ in the model. The $coverage$ is defined as the proportion of inputs which is not discarded\cite{street}. And we use "confidence threshold" to decide what to discard. For example, we set up a "confidence threhold" as 0.8 at first. Then any output vectors obtained from softmax unit would be discarded if the value of the maximum component is lower than 0.8.

Obviously, the value of $coverage$ will decrease if we set up a higher "confidence threhold". Therefore, at the beginning, we will set up a higher "confidence threshold" to help the model achieve a certain accuracy level. And when it performs well, we will increase the value of $coverage$ by reducing the "confidence threshold". \subsection{Experimemt results}






\section{Evaluation}



\section{Limited knowledge}



\section{Conclusion}




% use section* for acknowledgement
\section*{Acknowledgment}


This paper is supported by University of Southampton. I would like to thank Dr. for his guidance and patience during the entire work.





% trigger a \newpage just before the given reference
% number - used to balance the columns on the last page
% adjust value as needed - may need to be readjusted if
% the document is modified later
%\IEEEtriggeratref{8}
% The "triggered" command can be changed if desired:
%\IEEEtriggercmd{\enlargethispage{-5in}}

% references section

% can use a bibliography generated by BibTeX as a .bbl file
% BibTeX documentation can be easily obtained at:
% http://www.ctan.org/tex-archive/biblio/bibtex/contrib/doc/
% The IEEEtran BibTeX style support page is at:
% http://www.michaelshell.org/tex/ieeetran/bibtex/
%\bibliographystyle{IEEEtran}
% argument is your BibTeX string definitions and bibliography database(s)
%\bibliography{IEEEabrv,../bib/paper}
%
% <OR> manually copy in the resultant .bbl file
% set second argument of \begin to the number of references
% (used to reserve space for the reference number labels box)
\begin{thebibliography}{1}

\bibitem{disblief}
Dean, Jeffrey, et al. "Large scale distributed deep networks." Advances in neural information processing systems. 2012.



\bibitem{street}
Goodfellow, Ian J., et al. "Multi-digit number recognition from street view imagery using deep convolutional neural networks." arXiv preprint arXiv:1312.6082 (2013).
\bibitem{maxout}
Goodfellow, Ian J., et al. "Maxout networks." arXiv preprint arXiv:1302.4389 (2013).
\bibitem{image}
http://ufldl.stanford.edu/housenumbers/



\end{thebibliography}




% that's all folks
\end{document}


